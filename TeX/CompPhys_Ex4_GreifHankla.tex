\documentclass[12pt,a4paper,twoside]{article}

% Include packages that contain additional features, for example including special mathematical characters and images in your document
\usepackage{amssymb,amsmath,graphicx}
\usepackage[hidelinks]{hyperref}
\usepackage{verbatim}
%\usepackage{pdfpages}

\title{Exercises IV}
\author{Robin Greif (Exercise 3 Francisco Aros), Lia Hankla (Exercise 2 Victor Ksoll)}
\date{Due 2018/05/11}

\begin{document}
\maketitle

\section*{Exercise 2}
We first need to calculate the 12 equations that we need to solve. For a systems of 3-bodies with masses $m_1$, $m_2$, and $m_3$ and initial positions $\vec x_1$, $\vec x_2$, and $\vec x_3$, this is straightforward. Note that we restrict the motion to a plane so that $\vec x_i = x_{ix}\hat x + x_{iy}\hat y$.\\
\\
Defining separation vectors
\begin{align*}
  \vec r_{12} &= \vec x_2 - \vec x_1\\
  \vec r_{23} &= \vec x_3 - \vec x_2\\
  \vec r_{31} &= \vec x_1 - \vec x_3
\end{align*}
the equations of motion are (setting Newton's gravitational constant $G=1$ and using dot for time derivative):
\begin{align*}
  \ddot{\vec x}_1 &= \frac{m_2}{|r_{12}|^3}\vec r_{12}-\frac{m_3}{|r_{31}|^3}\vec r_{31}\\
  \ddot {\vec x}_2 &= \frac{m_3}{|r_{23}|^3}\vec r_{23}-\frac{m_1}{|r_{12}|^3}\vec r_{12}\\
  \ddot {\vec x}_3 &= \frac{m_1}{|r_{31}|^3}\vec r_{31}-\frac{m_2}{|r_{23}|^3}\vec r_{23}
\end{align*}
where $|r_{ij}|$ is the magnitude of the vector $\vec r_{ij}$.\\
\\
By introducing the variables $\vec v_{i} = \dot{\vec x}_i$, we have the full set of 12 equations (six times two components x/y each):
\begin{align*}
  \dot {\vec v}_1 &= \frac{m_2}{|r_{12}|^3}\vec r_{12}-\frac{m_3}{|r_{31}|^3}\vec r_{31}\\
  \dot {\vec v}_2 &= \frac{m_3}{|r_{23}|^3}\vec r_{23}-\frac{m_1}{|r_{12}|^3}\vec r_{12}\\
  \dot {\vec v}_3 &= \frac{m_1}{|r_{31}|^3}\vec r_{31}-\frac{m_2}{|r_{23}|^3}\vec r_{23}\\
  \dot {\vec x}_1 &= \vec v_1\\
  \dot {\vec x}_2 &= \vec v_2\\
  \dot {\vec x}_3 &= \vec v_3
\end{align*}
Now we can use the fourth-order Runge-Kutta integrator to plot orbits of the three bodies. As a test, we set the third body to be at the origin and have zero velocity so that the system acts as a two-body problem. The proof-of-concept is shown in Figure~\ref{fig:circ}.
\begin{figure}[h!]
  \centering
  \includegraphics[width=.9\textwidth]{figs/exercise03_0_stepsize001.pdf}
  \caption{Fourth-order Runge-Kutta proof-of-concept. Stars indicate starting location. Results are similar for other step sizes.} \label{fig:circ}
\end{figure}

\subsection*{Part a}
Simply plugging in the specificied initial conditions yields the plot in Figure~\ref{fig:2a}. Results are indistinguishable for step sizes between $0.01$ and 0.001.\\
\\
\begin{figure}[h!]
  \centering
  \includegraphics[width=.9\textwidth]{figs/exercise03_1_stepsize001.pdf}
  \caption{Fourth-order Runge-Kutta three-body integration. Stars indicate starting location. Results are similar for other step sizes.} \label{fig:2a}
\end{figure}


\subsection*{Part b: Trajectories}
Meissel-Burrau Problem using RK4

\begin{figure}[h]
  \centering
  \includegraphics[]{figs/exercise03_2_stepsize001_time20.pdf}
  \caption{Trajectories h = 0.001}
\end{figure}
\begin{figure}[h]
  \centering
  \includegraphics[]{figs/exercise03_2_stepsize00002_time9.pdf}
  \caption{Trajectories h = 0.0002}
\end{figure}
\begin{figure}[h]
   \centering
   \includegraphics[]{figs/exercise03_2_stepsize00001_time9.pdf}
   \caption{Trajectories h = 0.0001}
\end{figure}
\begin{figure}[h]
  \centering
  \includegraphics[]{figs/exercise03_2_stepsize5e-05_time10.pdf}
  \caption{Trajectories h = 0.00005}
\end{figure}[h!]

\subsection*{Part b: Distances}
\begin{figure}[h!]
  \centering
  \includegraphics[]{figs/exercise03_3_stepsize5e-05_time10.pdf}
  \caption{Distances, h = 0.00005}
\end{figure}[h!]

\subsection*{Part b: Total Energy Error}
\begin{figure}[h!]
  \centering
  \includegraphics[]{figs/exercise03_4_stepsize5e-05_time10.pdf}
  \caption{Error in Total Energy}
\end{figure}[h!]

%
The code for the exercises is as follows:
%\verbatiminput{C:/Users/Amelia/PycharmProjects/icp_exercises03/exercises03.py}
%
\end{document}

